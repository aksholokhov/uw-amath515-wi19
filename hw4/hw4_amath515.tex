\documentclass{article}

\usepackage{fancyhdr}
\usepackage{extramarks}
\usepackage{amsmath}
\usepackage{amsthm}
\usepackage{amsfonts}
\usepackage{tikz}
\usepackage[plain]{algorithm}
\usepackage{algpseudocode}

\usetikzlibrary{automata,positioning}

%
% Basic Document Settings
%

\usepackage{graphicx, url}
\graphicspath{figures/}

\topmargin=-0.45in
\evensidemargin=0in
\oddsidemargin=0in
\textwidth=6.5in
\textheight=9.0in
\headsep=0.25in

\linespread{1.1}

\pagestyle{fancy}
\lhead{\hmwkAuthorName}
\chead{\hmwkClass}
\rhead{\hmwkTitle}
\lfoot{\lastxmark}
\cfoot{\thepage}

\renewcommand\headrulewidth{0.4pt}
\renewcommand\footrulewidth{0.4pt}

\setlength\parindent{0pt}

%
% Create Problem Sections
%

\newcommand{\enterProblemHeader}[1]{
    \nobreak\extramarks{}{Problem \arabic{#1} continued on next page\ldots}\nobreak{}
    \nobreak\extramarks{Problem \arabic{#1} (continued)}{Problem \arabic{#1} continued on next page\ldots}\nobreak{}
}

\newcommand{\exitProblemHeader}[1]{
    \nobreak\extramarks{Problem \arabic{#1} (continued)}{Problem \arabic{#1} continued on next page\ldots}\nobreak{}
    \stepcounter{#1}
    \nobreak\extramarks{Problem \arabic{#1}}{}\nobreak{}
}

\setcounter{secnumdepth}{0}
\newcounter{partCounter}
\newcounter{homeworkProblemCounter}
\setcounter{homeworkProblemCounter}{1}
\nobreak\extramarks{Problem \arabic{homeworkProblemCounter}}{}\nobreak{}

%
% Homework Problem Environment
%
% This environment takes an optional argument. When given, it will adjust the
% problem counter. This is useful for when the problems given for your
% assignment aren't sequential. See the last 3 problems of this template for an
% example.
%
\newenvironment{homeworkProblem}[1][-1]{
    \ifnum#1>0
        \setcounter{homeworkProblemCounter}{#1}
    \fi
    \section{Problem \arabic{homeworkProblemCounter}}
    \setcounter{partCounter}{1}
    \enterProblemHeader{homeworkProblemCounter}
}{
    \exitProblemHeader{homeworkProblemCounter}
}

%
% Homework Details
%   - Title
%   - Due date
%   - Class
%   - Section/Time
%   - Instructor
%   - Author
%

\newcommand{\hmwkTitle}{Homework\ \#4}
\newcommand{\hmwkDueDate}{March 23, 2019}
\newcommand{\hmwkClass}{AMATH 515 (Optimization)}
\newcommand{\hmwkClassTime}{Section A}
\newcommand{\hmwkClassInstructor}{Prof. Alexander Aravkin}
\newcommand{\hmwkAuthorName}{\textbf{Alexey Sholokhov}}

%
% Title Page
%

\title{
    \vspace{2in}
    \textmd{\textbf{\hmwkClass:\ \hmwkTitle}}\\
    \normalsize\vspace{0.1in}\small{Due\ on\ \hmwkDueDate\ at 23:59}\\
    \vspace{0.1in}\large{\textit{\hmwkClassInstructor\ \hmwkClassTime}}
    \vspace{3in}
}

\author{\hmwkAuthorName}
\date{}

\renewcommand{\part}[1]{\textbf{\large Part \Alph{partCounter}}\stepcounter{partCounter}\\}

%
% Various Helper Commands
%

% Useful for algorithms
\newcommand{\alg}[1]{\textsc{\bfseries \footnotesize #1}}

% For derivatives
\newcommand{\deriv}[1]{\frac{\mathrm{d}}{\mathrm{d}x} (#1)}

% For partial derivatives
\newcommand{\pderiv}[2]{\frac{\partial}{\partial #1} (#2)}

% Integral dx
\newcommand{\dx}{\mathrm{d}x}

% Alias for the Solution section header
\newcommand{\solution}{\textbf{\large Solution}}

% Probability commands: Expectation, Variance, Covariance, Bias
\newcommand{\Var}{\mathrm{Var}}
\newcommand{\Cov}{\mathrm{Cov}}
\newcommand{\Bias}{\mathrm{Bias}}

% Different math operators
\DeclareMathOperator{\dom}{Dom}
\DeclareMathOperator{\diag}{Diag}
\DeclareMathOperator{\prox}{prox}
\DeclareMathOperator*{\proj}{proj}
\DeclareMathOperator*{\sign}{sign}

%% \mathbb symbols
\DeclareMathOperator{\E}{\mathbb{E}}
\DeclareMathOperator{\A}{\mathbb{A}}
\DeclareMathOperator{\R}{\mathbb{R}}
\DeclareMathOperator{\X}{\mathbb{X}}
\DeclareMathOperator{\N}{\mathbb{N}}
\DeclareMathOperator{\Q}{\mathbb{Q}}

%% \mathcal symbols
\DeclareMathOperator{\RR}{\mathcal{R}}
\DeclareMathOperator{\PP}{\mathcal{P}}
\DeclareMathOperator{\NN}{\mathcal{N}}
\DeclareMathOperator{\CC}{\mathcal{C}}



\begin{document}

\maketitle

\pagebreak

    
\begin{homeworkProblem}

Prove the following identity for $\alpha \in \mathbb{R}$:
\[
\|\alpha x + (1-\alpha) y\|^2 + \alpha(1-\alpha) \|x-y\|^2 = \alpha \|x\|^2 + (1-\alpha) \|y\|^2.
\]

\solution

\[
\begin{split}
    & \|\alpha x + (1 - \alpha)y\|^2 + \alpha(1 - \alpha)\|x - y\|^2 = \\
    = \,& \alpha\|x\|^2 + 2\alpha(1-\alpha)x^Ty + (1-\alpha)^2\|y\|^2 -2\alpha(1-\alpha)x^Ty + \alpha(1-\alpha)\|x\|^2 + \alpha(1-\alpha)\|y\|^2 = \\
    = \, & \alpha\|x\|^2 + (1-\alpha)\|y\|^2
\end{split}
\] \qed

\end{homeworkProblem}

\begin{homeworkProblem}
    An operator $T$ is {\it nonexpansive} if  $\|Tx - Ty\| \leq \|x - y\|$ for all $(x,y)$. 
For any such nonexpansive operator $T$, define 
\[
T_\lambda = (1-\lambda)I + \lambda T. 
\]
\begin{enumerate}
\item Show that $T_\lambda$ and $T$ have the same fixed points. 
\item Use problem 1 to show 
\[
\|T_\lambda z - \overline z\|^2 \leq \|z-\overline z\|^2 - \lambda(1-\lambda) \|z - Tz\|^2.
\]
where $\overline z$ is any fixed point of $T$, i.e. $T\overline z = \overline z$.

\end{enumerate}

\solution

\begin{enumerate}
    \item Let $x^*$ be a fixed point of $T$: $x^* = Tx^*$. Then for the operator $T_\lambda$ we have 
        \[
        T_\lambda x^* = (1 - \lambda)Ix^* + \lambda T x^* = (1 - \lambda)Tx^* + \lambda T x^* = Tx^* = x^*
        \] \qed
    \item Let $\bar{z}$ be $T\bar{z} =\bar{z}$
        \[
            \begin{split}
            & \|T_\lambda z - \bar{z}\|^2 \stackrel{a}{=} \|T_\lambda z - T_\lambda \bar{z}\|^2 = \|(1-\lambda)z + \lambda Tz - (1-\lambda)\bar{z} - \lambda T\bar{z}\|^2 = \\
            & \stackrel{1}{=} \lambda \|Tz - T\bar{z}\|^2 + (1-\lambda)\|z-\bar{z}\|^2 - \lambda(1-\lambda)\|Tz - T\bar{z} - z + \bar{z}\|^2 \leq \\
            & \leq (1-\lambda)\|z - \bar{z}\|^2 + \lambda\|z - \bar{z}\|^2 - \lambda(1 - \lambda)\|Tz - z \|^2 = \\
            & = \|z-\bar{z}\|^2 - \lambda(1-\lambda)\|z - Tz\|^2 
            \end{split}
        \] \qed
\end{enumerate}

\end{homeworkProblem}

\begin{homeworkProblem}
     An operator $T$ is {\it firmly nonexpansive} when it satisfies 
\[
\|Tx - Ty\|^2 + \|(I-T) x - (I-T)y\|^2 \leq \|x-y\|^2. 
\]


\begin{enumerate}
\item Show $T$ is firmly nonexpansive if and only if 
\[
\langle x-y, Tx - Ty \rangle \geq \|Tx - Ty\|^2. 
\]
\item Show $T$ is firmly nonexpansive if and only if 
\[
\langle Tx - Ty, (I-T)x - (I-T)y \rangle \geq 0. 
\]


\item Suppose that $S = 2T - I$. Let 
\[
\mu = \|Tx - Ty\|^2 + \|(I-T)x - (I-T)y\|^2 - \|x-y\|^2
\]
and let 
\[
\nu = \|Sx - Sy\|^2 - \|x-y\|^2.
\]
Show that $2\mu = \nu$ (you may find it helpful to use problem (1)). Conclude that 
$T$ is firmly nonexpansive exactly when $S$ is nonexpansive. 

\end{enumerate}

\solution

All the transitions below are equivalences (if and only if):

\begin{enumerate}
    \item 
        \[  \begin{split}
            & \|x - y\|^2 - \|x - y + (Tx - Ty)\|^2 = \\
            & = \|x - y\|^2 - \|x - y\|^2 + 2(x-y)^T(Tx - Ty) - \|Ty - Tx\|^2 \geq \|Tx - Ty\|^2 \iff \\
            & \iff (x-y)^T(Tx - Ty) \geq \|Tx - Ty\|^2
            \end{split} 
        \] \qed
    \item 
        \[
            \begin{split}
                 & (Tx - Ty)^T((I - T)x - (I - T)y) \geq 0 \iff \\
                 & \iff (Tx-Ty)^T((x-y) - (Tx - Ty)) \geq 0 \iff \\
                 & \iff (Tx-Ty)^T((x-y) \geq \|Tx - Ty\|^2 \stackrel{3.1}{\iff} \\
                 & \stackrel{3.1}{\iff} \|Tx - Ty\|^2 + \|(I-T) x - (I-T)y\|^2 \leq \|x-y\|^2
            \end{split} 
        \] \qed
    \item Let $S := 2T - I$, then we have
        \[
            \begin{split}
                & \nu + \|x - y\|^2 = \\
                & \|Sx - Sy\|^2 = \|(2I - T)x - (2I - T)y\|^2 = \|Tx - (I-T)x - Ty + (I-T)y\|^2 = \\
                & = \|Tx - Ty\|^2 + \|(I-T)x - (I-T)y\|^2 - 2(Tx - Ty)^T((I-T)x - (I-T)y = \\
                & = 3\|Tx - Ty\|^2 + \|(I-T)x - (I-T)y\|^2 - 2(Tx - Ty)^T(x-y) = \\
                & = 2\|Tx - Ty\|^2 + \|(I-T)x - (I-T)y\|^2 + \|Tx - Ty\|^2 - 2(Tx-Ty)^T(x-y) + \|x-y\|^2 - \|x - y\|^2 = \\
                & = 2\|Tx - Ty\|^2 + 2\|(I-T)x - (I-T)y\|^2 - \|x-y\|^2 = \\
                & = 2\mu + \|x-y\|^2 \iff \\ 
                & \iff \nu = 2\mu
            \end{split}
        \]
        The firm-nonexpansiveness of $T$ is equivalent to $\mu \leq 0$, the nonexpansiveness of $S$ is equivalent to $\nu \leq 0$, which can happen only simultaneously as $\mu = 2\nu$ \qed
\end{enumerate}


\end{homeworkProblem}

\textbf{The coding assignment is in .ipynb file attached to this homework}.

\end{document}
