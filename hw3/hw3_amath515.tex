\documentclass{article}

\usepackage{fancyhdr}
\usepackage{extramarks}
\usepackage{amsmath}
\usepackage{amsthm}
\usepackage{amsfonts}
\usepackage{tikz}
\usepackage[plain]{algorithm}
\usepackage{algpseudocode}

\usetikzlibrary{automata,positioning}

%
% Basic Document Settings
%

\usepackage{graphicx, url}
\graphicspath{figures/}

\topmargin=-0.45in
\evensidemargin=0in
\oddsidemargin=0in
\textwidth=6.5in
\textheight=9.0in
\headsep=0.25in

\linespread{1.1}

\pagestyle{fancy}
\lhead{\hmwkAuthorName}
\chead{\hmwkClass}
\rhead{\hmwkTitle}
\lfoot{\lastxmark}
\cfoot{\thepage}

\renewcommand\headrulewidth{0.4pt}
\renewcommand\footrulewidth{0.4pt}

\setlength\parindent{0pt}

%
% Create Problem Sections
%

\newcommand{\enterProblemHeader}[1]{
    \nobreak\extramarks{}{Problem \arabic{#1} continued on next page\ldots}\nobreak{}
    \nobreak\extramarks{Problem \arabic{#1} (continued)}{Problem \arabic{#1} continued on next page\ldots}\nobreak{}
}

\newcommand{\exitProblemHeader}[1]{
    \nobreak\extramarks{Problem \arabic{#1} (continued)}{Problem \arabic{#1} continued on next page\ldots}\nobreak{}
    \stepcounter{#1}
    \nobreak\extramarks{Problem \arabic{#1}}{}\nobreak{}
}

\setcounter{secnumdepth}{0}
\newcounter{partCounter}
\newcounter{homeworkProblemCounter}
\setcounter{homeworkProblemCounter}{1}
\nobreak\extramarks{Problem \arabic{homeworkProblemCounter}}{}\nobreak{}

%
% Homework Problem Environment
%
% This environment takes an optional argument. When given, it will adjust the
% problem counter. This is useful for when the problems given for your
% assignment aren't sequential. See the last 3 problems of this template for an
% example.
%
\newenvironment{homeworkProblem}[1][-1]{
    \ifnum#1>0
        \setcounter{homeworkProblemCounter}{#1}
    \fi
    \section{Problem \arabic{homeworkProblemCounter}}
    \setcounter{partCounter}{1}
    \enterProblemHeader{homeworkProblemCounter}
}{
    \exitProblemHeader{homeworkProblemCounter}
}

%
% Homework Details
%   - Title
%   - Due date
%   - Class
%   - Section/Time
%   - Instructor
%   - Author
%

\newcommand{\hmwkTitle}{Homework\ \#3}
\newcommand{\hmwkDueDate}{March 4, 2019}
\newcommand{\hmwkClass}{AMATH 515 (Optimization)}
\newcommand{\hmwkClassTime}{Section A}
\newcommand{\hmwkClassInstructor}{Prof. Alexander Aravkin}
\newcommand{\hmwkAuthorName}{\textbf{Alexey Sholokhov}}

%
% Title Page
%

\title{
    \vspace{2in}
    \textmd{\textbf{\hmwkClass:\ \hmwkTitle}}\\
    \normalsize\vspace{0.1in}\small{Due\ on\ \hmwkDueDate\ at 23:59}\\
    \vspace{0.1in}\large{\textit{\hmwkClassInstructor\ \hmwkClassTime}}
    \vspace{3in}
}

\author{\hmwkAuthorName}
\date{}

\renewcommand{\part}[1]{\textbf{\large Part \Alph{partCounter}}\stepcounter{partCounter}\\}

%
% Various Helper Commands
%

% Useful for algorithms
\newcommand{\alg}[1]{\textsc{\bfseries \footnotesize #1}}

% For derivatives
\newcommand{\deriv}[1]{\frac{\mathrm{d}}{\mathrm{d}x} (#1)}

% For partial derivatives
\newcommand{\pderiv}[2]{\frac{\partial}{\partial #1} (#2)}

% Integral dx
\newcommand{\dx}{\mathrm{d}x}

% Alias for the Solution section header
\newcommand{\solution}{\textbf{\large Solution}}

% Probability commands: Expectation, Variance, Covariance, Bias
\newcommand{\Var}{\mathrm{Var}}
\newcommand{\Cov}{\mathrm{Cov}}
\newcommand{\Bias}{\mathrm{Bias}}

% Different math operators
\DeclareMathOperator{\dom}{Dom}
\DeclareMathOperator{\diag}{Diag}
\DeclareMathOperator{\prox}{prox}
\DeclareMathOperator*{\proj}{proj}
\DeclareMathOperator*{\sign}{sign}

%% \mathbb symbols
\DeclareMathOperator{\E}{\mathbb{E}}
\DeclareMathOperator{\A}{\mathbb{A}}
\DeclareMathOperator{\R}{\mathbb{R}}
\DeclareMathOperator{\X}{\mathbb{X}}
\DeclareMathOperator{\N}{\mathbb{N}}
\DeclareMathOperator{\Q}{\mathbb{Q}}

%% \mathcal symbols
\DeclareMathOperator{\RR}{\mathcal{R}}
\DeclareMathOperator{\PP}{\mathcal{P}}
\DeclareMathOperator{\NN}{\mathcal{N}}
\DeclareMathOperator{\CC}{\mathcal{C}}



\begin{document}

\maketitle

\pagebreak

    
\begin{homeworkProblem}Compute the conjugates of the following functions.  
\begin{enumerate}
\item $f(x) = \delta_{\mathbb{B}_{\infty}}(x)$.
\item $f(x) = \delta_{\mathbb{B}_{2}}(x)$.
\item $f(x) = \exp(x)$.
\item $f(x) =  \log(1+\exp(x))$
\item $f(x) = x\log(x)$
\end{enumerate}
    
\solution
\begin{enumerate}
    \item $B_\infty = \{x\, |\: \max (|x_1|, \dots |x_n|) \leq 1\}$
     \[ f^*(z) = \sup_x\{ x^Tz - \delta_{B_\infty}(x)\} = \sup_{x \in B_\infty}x^Tz = \sum_{i=1}^n x_i*\sign(x_i) = \|x\|_1 \]
    \item $f(x) = \delta_{B_2}(x) \implies f^*(z) = \sup_{x \in B_2} z^Tx = \frac{z^Tz}{\|z\|_2} = \|z\|_2$
    \item $f(x) = \exp(x) \implies f^*(z) = \sup_x\{zx - \exp(x)\}$ \\
        \[
            \frac{\partial}{\partial x}(zx - \exp(x)) = z - exp(x) = 0 \implies x = \ln(z) 
        \]
        Substituting this value back
        \[
            f^*(z) = z\ln z - z 
        \]
    \item $f(x) = \log(1 + \exp(x)) \implies f^*(z) = \sup_x\{zx - \log(1 + \exp(x))\} = (*) $
        \[ 
            \frac{\partial}{\partial x}(zx - \log(1 + \exp(x))) = z - \frac{\exp(x)}{1 + \exp(x)} = 0 \implies x = \ln\frac{z}{1-z}
        \]
        
        $(*) = z\log \frac{z}{1-z} - \log\left(1+\frac{z}{1-z}\right) = z\log \frac{z}{1-z} - \log \frac{1}{1-z}  =\log(1-z)(1-z\log z)$
    \item $f(x) = x\log x \implies f^*(z) = \sup_x\{zx - x\log x\} = (*)$
        \[
            \begin{split}
            \frac{\partial}{\partial x} (zx - x\log x) =& z - (\log x + 1) = 0 \\
            x = & e^{z-1}
            \end{split}
        \] 
        $(*) = ze^{z-1} - (z-1)e^{z-1} = e^{z-1}$
        
        
\end{enumerate}

\end{homeworkProblem}


\begin{homeworkProblem}

Let $g$ be any convex function; $f$ is formed using $g$.
Compute $f^*$ in terms of $g^*$.  
\begin{enumerate}
\item $f(x) = \lambda g(x)$.
\item $f(x) = g(x-a) + \langle x, b \rangle$.
\item $f(x) = \inf_z \left\{g(x,z)\right\}$. 
\item $f(x) = \inf_z \left\{\frac{1}{2}\|x-z\|^2 + g(z)\right\}$
\end{enumerate}

\solution


\begin{enumerate}
    \item $f(x) = \lambda g(x)$
    \[
        f^*(z) = \sup_x \{ z^Tx - \lambda g(x)\} = \lambda \sup_x\left\{\left(\frac{1}{\lambda} z\right)^Tx - g(x)\right\} = \lambda g^*\left(\frac{1}{\lambda} z\right)
    \] \qed
    \item $f(x) = g(x-a) + x^Tb$
    \[
        \begin{split}
        f^*(z) = & \sup_x \{z^Tx - x^Tb - g(x-a)\} = \sup_x\{x^T(z-b) - g(x-a)\} = \sup_\xi \{(\xi + a)^T(z-b) - g(\xi)\} = \\
        = & a^T(z-b) + g^*(z-b)
        \end{split} 
    \]
    \qed
    \item $f(x) = \inf_z\{g(x, z)\} $
    \[
        f^*(k) = \sup_k\{k^Tx - \inf_z g(x, z)\} = \sup_{x, z}\{k^Tx + 0^Tz - g(x, z)\} = g^*(k, 0)
    \]
    \qed
    \item $f(x) = \inf_z \{\frac{1}{2}\|x - z\|^2 + g(z)\}$
    \[ 
    \begin{split}
        f^*(k) = & \sup_x\{k^Tx - \inf_z\{\frac{1}{2}\|x - z\|^2 + g(z)\}\} = \sup_x\{\sup_z \{k^Tx - \frac{1}{2}\|x - z\|^2 - g(z)\}\} = \\
        = & \sup_z\{\sup_x \{k^Tx - \frac{1}{2}\|x - z\|^2\} - g(z)\}
    \end{split}
    \]
    The internal $\sup_x$ is
    \[
        \frac{\partial }{\partial x}(k^Tx - \frac{1}{2}\|x - z\|^2) = k - (x - z) = 0 \implies x = k + z 
    \]
    substituting $x$ back:
    \[
        f^*(k) = \sup_z \{k^T(k+z) - \frac{1}{2}\|k\|^2 - g(z)\} = \sup_z \{k^Tz - g(z)\} + \frac{1}{2}\|k\|^2 = \frac{1}{2}\|k\|^2 + g^*(k) 
    \]
    \qed
    
\end{enumerate}


\end{homeworkProblem}

\begin{homeworkProblem}

Moreau Identities.
\begin{enumerate}
\item  Derive the Moreau Identity: 
\[
\mbox{prox}_{f}(z) + \mbox{prox}_{f^*}(z) = z. 
\]
You may find the `Fenchel flip' useful. 

\bigskip 
\item Use either of the Moreau identities and 
1a, 1b to check your formulas for 
\[
\mbox{prox}_{\|\cdot\|_1}, \quad \mbox{prox}_{\|\cdot\|_2}
\]
from last week's homework. 
\end{enumerate}

\solution

\begin{enumerate}
    \item Let $a(z) = \prox_f(z)$ and $b(z) = \prox_{f^*}(z)$. We need to prove that $z = a + b$ for all $z$. 
    \[
        a = \prox_f(z) = \arg \min_x\left\{\frac{1}{2}\|x - z\|^2 + f(x) \right\} \iff 0 \in (a - z) + \partial f(a)  
    \]
    We use the Fenchel flip
    \[
        z - a \in \partial f(a) \iff a \in \partial f^*(z-a)
    \]
    Next, we change variables:
    \[
        \text{Let } v = z-a \implies z-v \in \partial f^*(v)
    \]
    But by the definition of the proximal operator: 
    \[
        z - v \in \partial f^*(v) \implies v = \prox_{f^*}(z) \equiv b
    \]
    So we get
    \[
        b = z - a \implies z = a + b = \prox_f(z) + \prox_{f^*}(z)
    \] \qed
    
    \item \begin{enumerate}
        \item  According to the previous homework: $f(x) = \prox_{\|.\|_1}(x) = \begin{cases}
            x_i - 1 & x_i \geq 1 \\
            x_i + 1 &  x_i \leq -1 \\
            0 & x_i \in [-1, 1] 
        \end{cases}$\\
        According to the Moreau identities: $x = \prox_{\|.\|_1}(x) + \prox_{\|.\|_1^*}(x) \implies \prox_{\|.\|_1} = x - \prox_{\|.\|_1^*}(x)$
        \[
            \prox_{\|.\|_1^*}(x) = \prox_{\delta_{B_\infty}}(x) = \proj_{B_\infty}(x) = \begin{cases}
                x_i & x_i \in [-1, 1] \\
                1 & v_i \geq 1 \\
                -1 & v_i \leq -1
            \end{cases} = a(x)
        \]
        So 
        \[
            \prox_{\|.\|_1^*}(x) = x - a(x) = \begin{cases}
            x_i - 1 & x_i \geq 1 \\
            x_i + 1 &  x_i \leq -1 \\
            0 & x_i \in [-1, 1] 
        \end{cases}
        \]
        which matches to the previous homework's result. \qed
        
        \item According to the previous homework: $\prox_{\|.\|_2}(x) = x - \frac{x}{\|x\|_2}$.\\
        Using the Moreau identity: 
        \[
            x = \prox_{\|.\|_2}(x) + \prox_{\|.\|_2^*}(x) \implies \prox_{\|.\|_2}(x) = x - \prox_{\|.\|_2^*}(x)
        \]
        The conjugate function of the second norm is
        \[
            \|x\|_2^* = \sup_y\{x^Ty - \|y\|_2\}
        \]
        Using the optimality criterion
        \[
            \frac{\partial }{\partial y} \left(x^Ty - \|y\|_2\right) = x - \frac{y}{\|y\|_2} = 0 \implies y = \begin{cases}
                        x & \|x\|_2 = 1 \\
                        \text{does not exist} & \|x\|_2 \neq 1
                    \end{cases}
            \]
        The function $x^Ty - \|y\|_2$ is strictly concave, hence if its stationary points w.r.t. $y$ don't exist it implies that $sup_y\{x^Ty - \|y\|_2\} = \infty$. Consequently
        \[
            \|x\|_2^* = \delta_{\mathring B_2}(x) \text{ where } \mathring B_{2} = \{x\, |\: \|x\|_2 = 1\}
        \]
        So for the proximal operator of the conjugate we have
        \[
            \prox_{\|.\|_2^*}(x) = \proj_{\mathring B_2}(x) = \frac{x}{\|x\|_2}
        \]
        And for the proximal operator of the second norm
        \[
            \prox_{\|.\|_2}(x) = x - \prox_{\|.\|_2^*}(x) = x - \frac{x}{\|x\|_2}
        \]
        Which matches to the previous homework's result.  \qed
        \end{enumerate}
\end{enumerate}

\end{homeworkProblem}

\begin{homeworkProblem}

Duals of regularized GLM. Consider the Generalized Linear Model family: 
\[
\min_{x} \sum_{i=1}^n g(\langle a_i, x\rangle) - b^TAx + R(x),
\]
Where $g$ is convex and $R$ is any regularizer. 
\begin{enumerate}
\item Write down the general dual obtained from the perturbation 
\[
p(u) = \min_{x} \sum_{i=1}^n g(\langle a_i, x\rangle + u_i) - b^TAx + R(x).
\]
\bigskip\bigskip
\item Specify your formula to Ridge-regularized logistic regression: 
\[
\min_x \sum_{i=1}^n \log(1+\exp(\langle a_i, x \rangle))  - b^TAx  + \frac{\lambda}{2}\|x\|^2. 
\]
\bigskip\bigskip
\item Specify your formula to 1-norm regularized Poisson regression: 
\[
\min_x \sum_{i=1}^n \exp(\langle a_i, x \rangle) - b^TAx +  \lambda\|x\|_1. 
\]
\end{enumerate}

\solution

\begin{enumerate}
    \item According to the lecture material, the dual problem can be written using the perturbation approach as following: 
    \[
        sup_v -p^*(v) = sup_v-\phi^*(0, v)
    \]
    where $\phi(x, v)$ is defined as
    \[
    p(u) = \min_x \phi(x, u) = \min_x \sum_{i=1}^n g(\langle a_i, x\rangle + u_i) - b^TAx + R(x)
    \]
    We need to find the conjugate of $\phi(x, u)$
    \[
        \begin{split}
            \phi^*(z, v) = & \sup_{x, u} \{x^Tz + v^Tu - \sum_{i=1}^n g(\langle a_i, x\rangle + u_i) + b^TAx - R(x) \} = \\
            & \text{Let } w_i = a_i^Tx + u_i \implies w = Ax + u \implies u = w - Ax \\
            = & \sup_{x, w}\{x^T(z-A^Tb) + v^T(w - Ax) - \sum_{i=1}^n g(w_i) - R(x)\} \\
            = & \sup_{x, w}\{x^T(z + A^Tb - A^Tv) + v^Tw - \sum_{i=1}^n g(w_i) - R(x)\} = \\
            = & \sup_{x, w}\{x^T(z + A^Tb - A^Tv) - R(x) + \sum_{i=1}^n\left( v_iw_i - g(w_i) \right)\} = \\
            = & R^*(z + A^Tb - A^Tv) + \sum_{i=1}^n g^*(v_i)
        \end{split}
    \]
    
    So the dual problem is
    
    \[
        \sup_v - p^*(v) = \sup_v -\phi(0, v) = \sup_v -R^*(A^T(b-v)) - \sum_{i=1}^n g^*(v_i)
    \]
    
    \item 
    \[
    \min_x \sum_{i=1}^n \log(1+\exp(\langle a_i, x \rangle))  - b^TAx  + \frac{\lambda}{2}\|x\|^2. 
    \]
    Matching parts of this loss with our result from 4a:
    
    \[
        R^*(x) = \left(\frac{\lambda}{2}\|x\|_2^2\right) = \frac{\lambda}{2}\|x\|_2^2 \text{ -- as the second norm squared is self-conjugate }
    \]
    \[
        g(x) = \log(1+e^x) \implies g^*(z) = \log(1-z)(1 - z \log z) \text{ -- from 1.4}
    \]
    So the dual problem would be: 
    \[
        \sup_v - \frac{\lambda}{2}\|A^T(b-v)\|_2^2 - \sum_{i=1}^n \log(1-v_i)(1- v_i\log v_i)
    \]
    
    \item 
    \[
\min_x \sum_{i=1}^n \exp(\langle a_i, x \rangle) - b^TAx +  \lambda\|x\|_1. 
\]
    Matching parts of this loss with our result from 4a:
    \[
        R^*(x) = \left(\lambda\|.\|_1\right)^*(x) = \delta_{\lambda B_\infty}(x)
    \]
    \[
        g(x) = \exp(x) \implies g^*(z) = z\log z -z \text{ -- from 1.3}
    \]
    So the dual problem would be: 
    \[
        \begin{split}
        \sup_v & -\sum_{i=1}^nv_i\log v_i - v_i \\
        s.t.\: & A^T(b-v) \in \lambda B_{\infty}
        \end{split}
    \]
\end{enumerate}
    
\end{homeworkProblem}


\textbf{The coding assignment is in .ipynb file attached to this homework}.

\end{document}
